%` `%
\def\sectionabstract{产品简介}
%\section{概述}

%本手册主要提供HPM1000微控制器的功能介绍,主要特性和设计指标。\par
%本产品是一款超高性能通用微控制器,以RISC-V双核CPU为系统电源,主频达700 MHz,提供高达2 MB的片上SRAM存储器,丰富的外设:视频音频系统,信息安全模块,外部存储接口,定时器和高性能PWM,通信接口,模拟外设和完整的时钟和电源管理系统。本产品主要面向工业控制,智能家居,人机界面,马达控制,医疗电子,语音识别,数字信号处理和人工智能等多领域应用。

%\bookmark[rellevel=0, dest=secintro]{Test}
%\clearpage
%\hypertarget{secintro}{}%
%\bookmark[level=0, dest=secintro]{产品概述}
\section{简介} 
本文的要点如\autoref{tbl:appsum}:\par

\begin{center}    

    \begin{longtable}{|>{\raggedright\arraybackslash}m{1.5cm}
        |>{\raggedright\arraybackslash}m{10.5cm}
        |>{\raggedright\arraybackslash}m{3cm}|}
        \hline
		\textbf{编号}    & \textbf{功能描述}  & \textbf{备注} \\\endhead
	\hline
    1 & 功能1 & 1.0\\
    \hline
    2 & 功能2 & 1.0\\
    \hline    
    3 & 功能3  & 1.0\\
	\hline	
    \caption{功能描述简介}
    \label{tbl:appsum}
    \end{longtable}
\end{center}

%\begin{itemize}
%    \item RISC-V处理器本地存储器使用限制
%    \item 随机数发生器RNG使用限制
%    \item 显示接口LCDC场同步VSYNC中断/状态标志使用限制
%    \item 侵入检测模块TAMP使用限制
%    \item 模数转换器ADC12、ADC16的CONT\_EN控制位使用限制
%\end{itemize}


\clearpage
\subsection{功能1}
\subsubsection{功能1简介}
\begin{enumerate}
    \item ADC的介绍。  
    \item DMA的介绍,
\end{enumerate}\par
\par
\subsubsection{详细描述}
功能1详情:
\begin{enumerate}
    \item 介绍1。
    \item 介绍2。
    \item 介绍3。
\end{enumerate}
\par
\subsubsection{总结}
无。\par

\clearpage
\subsection{功能2}
\subsubsection{功能2简介}

\par
\subsubsection{总结}

\par


%\clearpage
%\subsection{E00014:USB Device IN endpoint使用限制}
%\subsubsection{问题描述}
%USB控制器配置为设备(Device)时,IN endpoint 4,5,6,7工作异常。\par
%\par
%\subsubsection{规避方法}
%USB控制器配置为设备(Device)时,建议用户使用IN endpoint 0,1,2,3。
%\par
%\subsubsection{计划修正}
%有。\par

%\clearpage
%\subsection{E00015:SYSCTL的CLOCK\_CPU寄存器写限制}
%\subsubsection{问题描述}
%对SYSCTL的CLOCK\_CPU寄存器写入后,其配置可能不生效。\par
%\par
%\subsubsection{规避方法}
%建议用户检查CLOCK\_CPU寄存器的DIV位域,如果CLOCK\_CPU寄存器待写入值的DIV位域相对原值DIV位域不变的,
%应当按以下步骤修改CLOCK\_CPU寄存器。\par
%\begin{enumerate}
%    \item 修改CLOCK\_CPU寄存器,仅修改DIV位域,建议DIV值稍大于原DIV值,如:DIV + 1。
%    \item 把待写入值写入CLOCK\_CPU寄存器。
%\end{enumerate}
%\par
%\subsubsection{修正情况}
%HPM6300系列微控制器版本2.0已修正。。\par

